\documentclass[a4paper,12pt]{article}
\usepackage[utf8]{inputenc}
\usepackage[spanish]{babel}
\usepackage{geometry}
\geometry{a4paper, margin=1in}
\usepackage{graphicx}
\usepackage{booktabs}
\usepackage{longtable}
\usepackage{caption}
\usepackage{xcolor}
\usepackage{colortbl}
\usepackage{amsmath}

% Preambulo para fuentes y compatibilidad
\usepackage{fontspec} % Omitir si usas PDFLaTeX
\usepackage{amsmath, amssymb}
\usepackage{fancyhdr}
\usepackage{lastpage}
\usepackage{hyperref}
\usepackage{listings}

% Configuración de página
\pagestyle{fancy}
\fancyhf{}
\fancyhead[L]{Informe de Dispositivos Móviles}
\fancyhead[R]{Fecha y Hora: 28 de agosto de 2025, 01:36 PM -03}
\fancyfoot[C]{Página \thepage\ de \pageref{LastPage}}

\begin{document}

\begin{center}
    {\Large \textbf{INFORME DE DISPOSITIVOS MÓVILES}} \\
    \vspace{0.2cm}
    Proyección de Stock y Demanda \\
    \vspace{0.2cm}
    Fecha de Reporte: 28 de agosto de 2025, 01:36 PM -03
\end{center}

\section*{Introducción}
Este informe tiene como objetivo estimar la cantidad de equipos celulares necesarios para cubrir las necesidades operativas durante el período \textit{junio-agosto 2025}. El análisis se basa en los datos recopilados por Mesa de Ayuda sobre los equipos entregados a las distribuidoras (EDEN, EDEA, EDELAP, EDES y EDESA) desde el 1 de junio de 2025, incluyendo: asignaciones a nuevos usuarios, recambios por robo, rotura, extravío u obsolescencia. Esta información permite establecer una proyección para el próximo trimestre, basada en el comportamiento real del período analizado.

\section{Distribuidoras}
A continuación, se presenta un resumen del comportamiento de consumo de equipos móviles por parte de cada distribuidora durante el trimestre analizado. Se incluyen datos de:
\begin{itemize}
    \item Solicitudes pendientes (resueltas o no según contexto)
    \item Nuevas asignaciones
    \item Recambios por obsolescencia, rotura, robo o extravío
\end{itemize}

\begin{figure}[h]
    \centering
    \includegraphics[width=0.8\textwidth]{images/tickets_pendientes.png}
    \caption{Análisis de Tickets Pendientes por Distribuidora (junio-agosto 2025)}
\end{figure}

\begin{longtable}{|l|c|c|c|c|p{5cm}|}
    \hline
    \rowcolor{gray!20} \textbf{Distribuidora} & \textbf{Stock Requerido} & \textbf{Stock Actual} & \textbf{Faltante} & \textbf{Prioridad} & \textbf{Acciones Sugeridas} \\ \hline
    DESA & 2 & 1 & -1 & \cellcolor{yellow!50} High & Reponer 1 unidades para cubrir demanda \\ 
    & & & & & Planificar reposición en próxima semana \\ \hline
    EDES & 2 & 1 & -1 & \cellcolor{yellow!50} High & Reponer 1 unidades para cubrir demanda \\ 
    & & & & & Planificar reposición en próxima semana \\ \hline
    EDELAP & 3 & 2 & -1 & \cellcolor{yellow!50} High & Reponer 1 unidades para cubrir demanda \\ 
    & & & & & Planificar reposición en próxima semana \\ \hline
    EDEN & 5 & 3 & -2 & \cellcolor{yellow!50} High & Reponer 2 unidades para cubrir demanda \\ 
    & & & & & Planificar reposición en próxima semana \\ \hline
    EDEA & 3 & 2 & -1 & \cellcolor{yellow!50} High & Reponer 1 unidades para cubrir demanda \\ 
    & & & & & Planificar reposición en próxima semana \\ \hline
    EDESA & 1 & 0 & -1 & \cellcolor{red!50} Critical & \textbf{URGENTE}: Stock crítico - Coordinar reposición inmediata \\ \hline
    \caption{Tabla de Análisis de Stock por Distribuidora}
\end{longtable}

\section{Análisis}
\subsection{Teléfonos Pendientes de Recambio (Solicitud y Obsolescencia)}
Actualmente, no se registran tickets pendientes activos. No obstante, en la base de datos de SOTI MobiControl se identifican \textbf{26} dispositivos que se encuentran por debajo del estándar vigente (Samsung Galaxy A16/A2X).

\begin{figure}[h]
    \centering
    \includegraphics[width=0.8\textwidth]{images/issues_frecuentes.png}
    \caption{Issues Más Frecuentes (junio-agosto 2025)}
\end{figure}

\subsection{Teléfonos Reemplazados}
Durante el trimestre junio-agosto 2025, se entregaron \textbf{64} dispositivos entre asignaciones nuevas y recambios.

\begin{figure}[h]
    \centering
    \includegraphics[width=0.8\textwidth]{images/distribucion_tipos_issues.png}
    \caption{Distribución por Distribuidoras y Tipos de Issues (junio-agosto 2025)}
\end{figure}

\section{Conclusión}
En el último trimestre (junio-agosto 2025), se entregaron 64 equipos celulares (nuevos y recambios por robo, extravío, rotura u obsolescencia). Comparado con el trimestre anterior (mayo-julio 2025, 73 equipos), la demanda se mantuvo estable.
\begin{itemize}
    \item Distribución de entregas:
    \begin{itemize}
        \item Nuevos ingresos: 20 unidades
        \item Rotura: 15 unidades
        \item Robo/extravío: 10 unidades
        \item Obsolescencia: 19 unidades
    \end{itemize}
\end{itemize}

\subsection{Proyección Próximo Trimestre (septiembre-noviembre 2025)}
\begin{itemize}
    \item Demanda estimada: 56 equipos (basado en tendencia).
    \item Equipos obsoletos (SOTI): 26 unidades.
    \item Total estimado: 56 + 26 = \textbf{82} equipos.
\end{itemize}

\begin{figure}[h]
    \centering
    \includegraphics[width=0.8\textwidth]{images/proyecciones_demanda.png}
    \caption{Proyecciones de Demanda por Distribuidora (septiembre-noviembre 2025)}
\end{figure}

\subsubsection{Presupuesto}
\begin{itemize}
    \item Modelo: Samsung Galaxy A16 (U$S 576/unidad).
    \item Costo total: 82 × U$S 576 = U$S \textbf{47,232}.
\end{itemize}

\subsection{Nota}
Aunque actualmente no hay tickets pendientes activos, esta estimación contempla posibles ingresos y recambios durante los próximos meses, con base en la evolución reciente de la demanda.

\section{Estado Actual de Stock de Dispositivos}
El stock actual disponible se compone de 131 dispositivos, de los cuales:
\begin{itemize}
    \item 117 unidades del modelo Galaxy A16 están destinadas a cubrir la demanda operativa general (asignaciones, recambios y reemplazos).
    \item El resto del stock corresponde a modelos de gama media y alta, utilizados en situaciones específicas para jefaturas, coordinaciones o gerencias.
\end{itemize}
Este nivel de stock permite, por ahora, responder tanto a la demanda proyectada como a los equipos obsoletos detectados. Sin embargo, en caso de un pico de demanda inesperado o de una demora en la reposición por parte del proveedor, se recomienda reforzar la reserva operativa con una compra preventiva de al menos 30 unidades.

\section{Escenario Mínimo y Urgente}
En caso de que se priorice solo cubrir lo ya solicitado a través de tickets aprobados, la demanda mínima a cubrir es de \textbf{6} dispositivos (suma de faltantes críticos).
Este escenario contempla únicamente los reemplazos ya solicitados, que se distribuyen del siguiente modo:
\begin{itemize}
    \item EDEA: 1
    \item EDELAP: 1
    \item EDESA: 1
    \item EDEN: 2
    \item EDES: 1
    \item DESA: 0
\end{itemize}
Presupuesto estimado: U$S \textbf{3,456} (6 × U$S 576).

\section{Adquisición de Accesorios}
Para acompañar los dispositivos mencionados anteriormente, se recomienda también la compra de los siguientes accesorios, que cubrirán tanto la entrega inmediata como posibles solicitudes por recambios:
\begin{itemize}
    \item 50 vidrios templados para Galaxy A25
    \item 50 fundas para Galaxy A25
    \item 30 cabezales de cargador tipo C
\end{itemize}
Presupuesto estimado para accesorios: U$S \textbf{1,500}.

\section{Proyección de Gama Media / Alta}
En paralelo al recambio convencional, se prevé la necesidad de reemplazar o asignar nuevos equipos para personal de supervisión, gerencia o dirección.
Se estima el siguiente requerimiento:
\begin{itemize}
    \item Galaxy A35: 5 unidades
    \item Galaxy A55: 3 unidades
    \item Galaxy S25+: 2 unidades
\end{itemize}
Presupuesto estimado: U$S \textbf{5,760} (Incluye equipos y accesorios correspondientes).

\begin{figure}[h]
    \centering
    \includegraphics[width=0.8\textwidth]{images/tendencia_temporal.png}
    \caption{Tendencia Temporal de Demanda (junio-agosto 2025)}
\end{figure}

\end{document}